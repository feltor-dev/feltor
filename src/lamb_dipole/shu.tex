%%%%%%%%%%%%%%%%%%%%%definitions%%%%%%%%%%%%%%%%%%%%%%%%%%%%%%%%%%%%%%%

%\documentclass[12pt]{article}
%\documentclass[12pt]{scrartcl}
\documentclass{hitec} % contained in texlive-latex-extra
\settextfraction{0.9} % indent text
\usepackage{csquotes}
\usepackage[hidelinks]{hyperref} % doi links are short and usefull?
\hypersetup{%
    colorlinks=true,
    linkcolor=blue,
    urlcolor=magenta
}
\urlstyle{rm}
\usepackage[english]{babel}
\usepackage{mathtools} % loads and extends amsmath
\usepackage{amssymb}
% packages not used
%\usepackage{graphicx}
%\usepackage{amsthm}
%\usepackage{subfig}
\usepackage{bm}
\usepackage{longtable}
\usepackage{booktabs}
\usepackage{ragged2e} % maybe use \RaggedRight for tables and literature?
\usepackage[table]{xcolor} % for alternating colors
%\rowcolors{2}{gray!25}{white}
\renewcommand\arraystretch{1.3}
\usepackage{doi}
\usepackage[sort,square,numbers]{natbib}
\bibliographystyle{abbrvnat}

\definecolor{light-gray}{gray}{0.95}
\newcommand{\code}[1]{\colorbox{light-gray}{\texttt{#1}}}
\newcommand{\eps}{\varepsilon}
\renewcommand{\d}{\mathrm{d}}
\renewcommand{\vec}[1]{{\mathbf{#1}}}
\newcommand{\dx}{\,\mathrm{d}x}
%\newcommand{\dA}{\,\mathrm{d}(x,y)}
%\newcommand{\dV}{\mathrm{d}^3{x}\,}
\newcommand{\dA}{\,\mathrm{dA}}
\newcommand{\dV}{\mathrm{dV}\,}

\newcommand{\Eins}{\mathbf{1}}

\newcommand{\ExB}{$\bm{E}\times\bm{B} \,$}
\newcommand{\GKI}{\int d^6 \bm{Z} \BSP}	
\newcommand{\GKIV}{\int dv_{\|} d \mu d \theta \BSP}	
\newcommand{\BSP}{B_{\|}^*}
\newcommand{\GA}[1]{\langle #1	 \rangle}

\newcommand{\Abar}{\langle A_\parallel \rangle}
%Vectors
\newcommand{\bhat}{\bm{\hat{b}}}
\newcommand{\bbar}{\overline{\bm{b}}}
\newcommand{\chat}{\bm{\hat{c}}}
\newcommand{\ahat}{\bm{\hat{a}}}
\newcommand{\xhat}{\bm{\hat{x}}}
\newcommand{\yhat}{\bm{\hat{y}}}
\newcommand{\zhat}{\bm{\hat{z}}}

\newcommand{\Xbar}{\bar{\vec{X}}}
\newcommand{\phat}{\bm{\hat{\perp}}}
\newcommand{\that}{\bm{\hat{\theta}}}

\newcommand{\eI}{\bm{\hat{e}}_1}
\newcommand{\eII}{\bm{\hat{e}}_2}
\newcommand{\ud}{\mathrm{d}}

%Derivatives etc.
\newcommand{\pfrac}[2]{\frac{\partial#1}{\partial#2}}
\newcommand{\ffrac}[2]{\frac{\delta#1}{\delta#2}}
\newcommand{\fixd}[1]{\Big{\arrowvert}_{#1}}
\newcommand{\curl}[1]{\nabla \times #1}
\newcommand{\np}{\nabla_{\perp}}
\newcommand{\npc}{\nabla_{\perp} \cdot }
\newcommand{\nc}{\nabla\cdot }
\newcommand{\GAI}{\Gamma_{1}^{\dagger}}
\newcommand{\GAII}{\Gamma_{1}^{\dagger -1}}
\newcommand{\T}{\mathrm{T}}
\newcommand{\Tp}{\mathcal T^+_{\Delta\varphi}}
\newcommand{\Tm}{\mathcal T^-_{\Delta\varphi}}
\newcommand{\Tpm}{\mathcal T^\pm_{\Delta\varphi}}


%%%%%%%%%%%%%%%%%%%%%%%%%%%%%DOCUMENT%%%%%%%%%%%%%%%%%%%%%%%%%%%%%%%%%%%%%%%
\begin{document}

\title{Testing Advection Schemes}
\author{ M.~Wiesenberger}
\maketitle

\begin{abstract}
  This is a program to test various advection schemes on the 2d incompressible Euler
  equation used in Reference~\cite{Einkemmer2014}.
\end{abstract}

\section{Equations}
We implement the 2d incompressible Euler equation
\begin{subequations}
\begin{align}
 \frac{\partial \omega}{\partial t} + \{ \phi, \omega\} = 0 \\
 -\Delta \phi = \omega
\end{align}
\label{eq:euler_poisson}
\end{subequations}
with vorticity $\omega$ and stream-function $\phi$.
The Poisson bracket is given by $\{ \phi, \omega\} := \phi_x \omega_y - \phi_y \omega_x$.
Eq.~\eqref{eq:euler_poisson} is a reformulation of the standard conservative form
\begin{subequations}
\begin{align}
    \frac{\partial \omega}{\partial t} + \nabla\cdot({\vec v \omega}) = 0 \\
\nabla\cdot\vec v = 0 \quad \omega = -(\nabla\times \vec v)\cdot \zhat
\end{align}
\label{eq:euler_conservative}
\end{subequations}
with $v_x = - \phi_y$ and $v_y = \phi_x$.

Eqs.~\eqref{eq:euler_poisson} have an infinite amount of conserved quantities
among them the total vorticity $V$, the kinetic energy $E$ and the enstrophy $\Omega$
 \begin{align}
     V := \int_D \omega \dA\quad
     E :=\frac{1}{2} \int_D \left( \nabla \phi\right)^2 \dA \quad
     \Omega:= \frac{1}{2} \int_D \omega^2 \dA
 \end{align}


\section{Initialization}
We will consider several different initial conditions in order to test
our numerical methods
\subsection{Lamb Dipole}
The Lamb dipole is a stationary solution to the Euler equations~\cite{Nielsen1997} with infinite
boundary conditions
\begin{align}
    \omega(x,y,0) = \begin{cases}
        \frac{2\lambda U}{J_0(\lambda R)} J_1(\lambda R) \cos \theta,\ r < R,\\
        0, \text{ else}
    \end{cases}
\end{align}
Unfortunately, for a finite box this is not an exact solution any more.
on the domain $[0,1]\times [0,1]$.
\subsection{Manufactured Solution}
We manufacture a solution via
\begin{align}
    \phi(x,y,t) &=
    x \exp\left( - \frac{ x^2 + (y+vt)^2}{\sigma^2}\right) \\
    \omega(x,y,t) &= -\Delta \phi = -\sigma^{-4} \left[ 4\phi(x,y,t) ( x^2-2\sigma^2  + (y+tv)^2)\right]
\end{align}
which is solution to the modified equations
\begin{subequations}
\begin{align}
    \frac{\partial \omega}{\partial t} + \{ \phi, \omega\} = S(x,y,t) \\
 -\Delta \phi = \omega
\end{align}
\label{eq:euler_poisson}
\end{subequations}
with the source
\begin{align}
    S(x,y,t) = 8 x \sigma^{-6}(y+vt)\exp\left( - 2\frac{ x^2 + (y+vt)^2}{\sigma^2} \right)\left(-\sigma^2  + \exp\left( \frac{ x^2 + (y+vt)^2}{\sigma^2} \right) v( -3\sigma^2 + x^2 + (y+vt)^2) \right)
\end{align}
on the domain $[-1,1]\times [-1,1]$.


\subsection{ Double Shear layer}
Here, we follow~\cite{Liu2000} and test the scheme on a double shear layer problem.
\begin{align}
    \omega(x,y,0) = \begin{cases}
        \delta \cos(x) - \frac{1}{\rho} \text{sech}^2 \left(\frac{y-\pi/2}{\rho}\right),\ y \leq \pi \\
        \delta \cos(x) + \frac{1}{\rho} \text{sech}^2 \left(\frac{3\pi/2-y}{\rho}\right),\ y > \pi \\
    \end{cases}
\end{align}
where $\rho = \pi/15$ and $\delta =0.05$ on the domain $[0,2\pi]\times [0,2\pi]$.
This solution will quickly roll-up and generate smaller and smaller scales.
A thin shear layer corresponds to $\rho = \pi/50$ or smaller.

\section{Numerical methods}
Our goal is to try out various time integration and advection discretization techniques.
We know from Godunov's theorem
that any linear advection scheme of order 2 or higher is prone to oscillations.
\subsection{Forward time and centered space}
It is well known that the forward in time, centered in space method for solving
hyperbolic systems is unconditionally unstable~\cite{LeVeque}.
\subsection{Arakawa scheme and centered flux}
Reference~\cite{Liu2000} reports that the centered flux does not have any numerical
diffusion while the upwind flux does.
They also prove that upwind and centered fluxes do not dissipate energy.
From finite differences we know that centered differences for the advection term
are unstable (or at least produce a lot of oscillations).
\subsection{Modal filters}
\ldots
\subsection{Artificial viscosity}
\ldots

\section{Compilation and useage}
The program shu\_b.cu compiles with
\begin{verbatim}
make <shu_b> device = <omp gpu>
make <shu_hpc> device = <omp gpu>
\end{verbatim}
and depends on both GLFW3 and NETCDF. If GLFW3 is not available then compile shu\_hpc which avoids this dependency.
Run with
\begin{verbatim}
path/to/feltor/src/shu/shu_b input.json
\end{verbatim}

\subsection{Input file structure}
Input file format: json

%%This is a booktabs table
\begin{longtable}{llll>{\RaggedRight}p{7cm}}
\toprule
\rowcolor{gray!50}\textbf{Name} &  \textbf{Type} & \textbf{Example} & \textbf{Default} & \textbf{Description}  \\ \midrule
\bottomrule
\end{longtable}

The default value is taken if the value name is not found in the input file. If there is no default and
the value is not found,
the program exits with an error message.

\subsection{Structure of output file}
Output file format: netcdf-4/hdf5
%
%Name | Type | Dimensionality | Description
%---|---|---|---|
\begin{longtable}{lll>{\RaggedRight}p{7cm}}
\toprule
\rowcolor{gray!50}\textbf{Name} &  \textbf{Type} & \textbf{Dimension} & \textbf{Description}  \\ \midrule
inputfile  &             text attribute & 1 & verbose input file as a string \\
energy\_time             & Coord. Var. & 1 (energy\_time) & timesteps at which 1d variables are written \\
time                     & Coord. Var. & 1 (time) & time at which fields are written \\
x                        & Coord. Var. & 1 (x) & x-coordinate  \\
y                        & Coord. Var. & 1 (y) & y-coordinate \\
vorticity\_field         & Dataset & 3 (time, y, x) & electon density $n$ \\
potential                & Dataset & 3 (time, y, x) & electric potential $\phi$  \\
vorticity                & Dataset & 1 (energy\_time) & Vorticity $V$  \\
enstrophy                & Dataset & 1 (energy\_time) & Enstropy $\Omega$  \\
energy                   & Dataset & 1 (energy\_time) & Total energy integral computed using $E = \int_D \phi\omega \dA$ \\
variation                & Dataset & 1 (energy\_time) & Total energy integral computed directly $\int_D (\nabla \phi)^2$  \\
\bottomrule
\end{longtable}

%..................................................................
\bibliography{../../doc/related_pages/references}
%..................................................................


\end{document}
