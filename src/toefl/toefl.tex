%\documentclass[12pt]{article}
%\documentclass[12pt]{scrartcl}
\documentclass{hitec}
\usepackage{graphicx}
\usepackage{amsmath}
\usepackage{amsfonts}
\usepackage{amssymb}
\usepackage{amsthm}
\usepackage{subfig}
\usepackage{bm}
\usepackage{longtable}
\usepackage{booktabs}
\usepackage[table]{xcolor} % for alternating colors
\rowcolors{2}{gray!25}{white}
\renewcommand\arraystretch{1.3}

%%%%%%%%%%%%%%%%%%%%%definitions%%%%%%%%%%%%%%%%%%%%%%%%%%%%%%%%%%%%%%%

\newcommand{\eps}{\varepsilon}
\renewcommand{\d}{\mathrm{d}}
\newcommand{\T}{\mathrm{T}}
\renewcommand{\vec}[1]{{\mathbf{#1}}}
\newcommand{\dx}{\,\mathrm{d}x}
%\newcommand{\dA}{\,\mathrm{d}(x,y)}
%\newcommand{\dV}{\mathrm{d}^3{x}\,}
\newcommand{\dA}{\,\mathrm{dA}}
\newcommand{\dV}{\mathrm{dV}\,}

\newcommand{\Eins}{\mathbf{1}}

\newcommand{\ExB}{$\bm{E}\times\bm{B} \,$}
\newcommand{\GKI}{\int d^6 \bm{Z} \BSP}	
\newcommand{\GKIV}{\int dv_{\|} d \mu d \theta \BSP}	
\newcommand{\BSP}{B_{\|}^*}
\newcommand{\GA}[1]{\langle #1	 \rangle}

\newcommand{\Abar}{\langle A_\parallel \rangle}
%Vectors
\newcommand{\bhat}{\bm{\hat{b}}}
\newcommand{\bbar}{\overline{\bm{b}}}
\newcommand{\chat}{\bm{\hat{c}}}
\newcommand{\ahat}{\bm{\hat{a}}}
\newcommand{\xhat}{\bm{\hat{x}}}
\newcommand{\yhat}{\bm{\hat{y}}}
\newcommand{\zhat}{\bm{\hat{z}}}

\newcommand{\Xbar}{\bar{\vec{X}}}
\newcommand{\phat}{\bm{\hat{\perp}}}
\newcommand{\that}{\bm{\hat{\theta}}}

\newcommand{\eI}{\bm{\hat{e}}_1}
\newcommand{\eII}{\bm{\hat{e}}_2}
\newcommand{\ud}{\mathrm{d}}

%Derivatives etc.
\newcommand{\pfrac}[2]{\frac{\partial#1}{\partial#2}}
\newcommand{\ffrac}[2]{\frac{\delta#1}{\delta#2}}
\newcommand{\fixd}[1]{\Big{\arrowvert}_{#1}}
\newcommand{\curl}[1]{\nabla \times #1}
\newcommand{\np}{\nabla_{\perp}}
\newcommand{\npc}{\nabla_{\perp} \cdot }
\newcommand{\nc}{\nabla\cdot }
\newcommand{\GAI}{\Gamma_{1}^{\dagger}}
\newcommand{\GAII}{\Gamma_{1}^{\dagger -1}}

%%%%%%%%%%%%%%%%%%%%%%%%%%%%%DOCUMENT%%%%%%%%%%%%%%%%%%%%%%%%%%%%%%%%%%%%%%%
\begin{document}

\title{The toefl project }
\maketitle
 
\begin{abstract}
This is a program for 2d isothermal blob simulations.
\end{abstract}

\section{Equations}
Currently you can choose between $5$ slightly different sets of equations
\begin{align}
 \frac{\partial n}{\partial t} + \nabla\cdot\left( n \vec u_E  \right) &= 0  \\
    \frac{\partial \rho}{\partial t} + \nabla\cdot\left( \rho\vec u_E \right) + \nabla \cdot\left( n(\vec u_\psi + \vec u_d + \vec u_g) \right) &= 0\\
\rho = \nabla\cdot( n\nabla_\perp \phi / \Omega B)
\end{align}
...
\section{Structure of input file}
File format: json

%%This is a booktabs table
\begin{longtable}{llllp{7cm}}
\toprule
\rowcolor{gray!50}\bf Name &  \bf Type & \bf Example & \bf Default & \bf Description  \\ \midrule
n      & integer & 3 & - &\# x-y-polynomials \\
Nx     & integer &100& - &\# grid points in x \\
Ny     & integer &100& - &\# grid points in y \\
dt     & integer &3.0& - &time step in units of c\_s/rho\_s \\
n\_out  & integer &3  & - &\# x-y polynomials in output \\
Nx\_out & integer &100& - &\# grid points in x in output fields \\
Ny\_out & integer &100& - &\# grid points in y in output fields \\
itstp  & integer &2  & - &   steps between outputs \\
maxout & integer &100& - &      \# outputs excluding first \\
eps\_pol   & float &1e-6    & - &  accuracy of polarisation solver \\
eps\_gamma & float &1e-7    & - & accuracy of Gamma  \\
eps\_time  & float &1e-10   & - & accuracy of implicit time-stepper \\
curvature  & float &0.00015& - & magnetic curvature \\
tau        & float &1      & - & Ti/Te (only in gyrofluid models) \\
nu\_perp    & float &5e-3   & - & pependicular viscosity \\
amplitude  & float &1.0    & - & amplitude of the blob \\
sigma      & float &10     & - & blob variance in units of rho\_s \\
posX       & float &0.3    & - & blob x-position in units of lx \\
posY       & float &0.5    & - & blob y-position in units of ly \\
lx         & float &200    & - & lx in units of rho\_s \\
ly         & float &200    & - & ly in units of rho\_s \\
friction   & float & 0     & 0 & friction coefficient in gravity model \\
bc\_x   & char & "DIR"      & - & boundary condition in x (one of PER, DIR, NEU, DIR\_NEU or NEU\_DIR) \\
bc\_y   & char & "PER"      & - & boundary condition in y (one of PER, DIR, NEU, DIR\_NEU or NEU\_DIR) \\
equations  & char & "global" & "global" &local, global, gravity\_local, gravity\_global, drift\_global \\
boussinesq & bool & false    & false &boussinesq approximation in global models true or false\\
\bottomrule
%\end{tabular}
\end{longtable}


The default value is taken if the value name is not found in the input file. If there is no default and 
the value is not found, 
the program exits with an error message. 




\section{Structure of output file}
File Format: netcdf-4 /hdf5
%
%Name | Type | Dimensionality | Description 
%---|---|---|---|
\begin{longtable}{lllp{8cm}}
\toprule
\rowcolor{gray!50}\bf Name &  \bf Type & \bf Dimension & \bf Description  \\ \midrule
inputfile  &             text attribute & 1 & verbose input file as a string \\ 
energy\_time             & Dataset & 1 & timesteps at which 1d variables are written \\
time                     & Dataset & 1 & time at which fields are written \\
x                        & Dataset & 1 & x-coordinate  \\
y                        & Dataset & 1 & y-coordinate \\
electrons                & Dataset & 3 & electon density (time, y, x) \\
ions                     & Dataset & 3 & ion density (time, y, x) \\
potential                & Dataset & 3 & electric potential (time, y, x) \\
vorticity                & Dataset & 3 & Laplacian of potential (time, y, x) \\
dEdt                     & Dataset & 1 & change of energy per time (energy\_time) \\
dissipation              & Dataset & 1 & diffusion integrals (energy\_time) \\
energy                   & Dataset & 1 & total energy integral (energy\_time) \\
mass                     & Dataset & 1 & mass integral (energy\_time)  \\
\bottomrule
%\end{tabular}
\end{longtable}
%..................................................................
\begin{thebibliography}{1}
    \bibitem{Wiesenberger2017}
    M. Wiesenberger, M. Held, R. Kube and O.E. Garcia, Phys. Plasmas { 24}, 064502 (2017)
    \bibitem{Kube2016} \vspace{-8pt} % all references except the first must have this vspace correction
    R. Kube, O.E. Garcia, and M. Wiesenberger, Phys. Plasmas {23}, 122303 (2016)
    \bibitem{Wiesenberger2014} \vspace{-8pt} % all references except the first must have this vspace correction
    M. Wiesenberger, J. Madsen, and A. Kendl, "Radial convection of finite ion temperature, high amplitude plasma blobs", Phys. Plasmas { 21}, 092301 (2014) (https://doi.org/10.1063/1.4894220 )
\end{thebibliography}



%..................................................................


\end{document}

