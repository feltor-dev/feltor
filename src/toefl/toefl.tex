%%%%%%%%%%%%%%%%%%%%%definitions%%%%%%%%%%%%%%%%%%%%%%%%%%%%%%%%%%%%%%%

%\documentclass[12pt]{article}
%\documentclass[12pt]{scrartcl}
\documentclass{hitec} % contained in texlive-latex-extra
\settextfraction{0.9} % indent text
\usepackage{csquotes}
\usepackage[hidelinks]{hyperref} % doi links are short and usefull?
\hypersetup{%
    colorlinks=true,
    linkcolor=blue,
    urlcolor=magenta
}
\urlstyle{rm}
\usepackage[english]{babel}
\usepackage{mathtools} % loads and extends amsmath
\usepackage{amssymb}
% packages not used
%\usepackage{graphicx}
%\usepackage{amsthm}
%\usepackage{subfig}
\usepackage{bm}
\usepackage{longtable}
\usepackage{booktabs}
\usepackage{ragged2e} % maybe use \RaggedRight for tables and literature?
\usepackage[table]{xcolor} % for alternating colors
%\rowcolors{2}{gray!25}{white}
\renewcommand\arraystretch{1.3}
\usepackage{doi}
\usepackage[sort,square,numbers]{natbib}
\bibliographystyle{abbrvnat}

\definecolor{light-gray}{gray}{0.95}
\newcommand{\code}[1]{\colorbox{light-gray}{\texttt{#1}}}
\newcommand{\eps}{\varepsilon}
\renewcommand{\d}{\mathrm{d}}
\renewcommand{\vec}[1]{{\boldsymbol{#1}}}
\newcommand{\dx}{\,\mathrm{d}x}
%\newcommand{\dA}{\,\mathrm{d}(x,y)}
%\newcommand{\dV}{\mathrm{d}^3{x}\,}
\newcommand{\dA}{\,\mathrm{dA}}
\newcommand{\dV}{\mathrm{dV}\,}

\newcommand{\Eins}{\mathbf{1}}

\newcommand{\ExB}{$\bm{E}\times\bm{B} \,$}
\newcommand{\GKI}{\int d^6 \bm{Z} \BSP}
\newcommand{\GKIV}{\int dv_{\|} d \mu d \theta \BSP}
\newcommand{\BSP}{B_{\|}^*}
\newcommand{\Abar}{\langle A_\parallel \rangle}
%Averages
\newcommand{\RA}[1]{\left \langle #1 \right \rangle} %Reynolds (flux-surface) average
\newcommand{\RF}[1]{\widetilde{#1}} %Reynolds fluctuation
\newcommand{\FA}[1]{\left[\left[ #1 \right]\right]} %Favre average
\newcommand{\FF}[1]{\widehat{#1}} %Favre fluctuation
\newcommand{\PA}[1]{\left \langle #1 \right\rangle_\varphi} %Phi average

%Vectors
\newcommand{\ahat}{\bm{\hat{a}}}
\newcommand{\bhat}{\bm{\hat{b}}}
\newcommand{\chat}{\bm{\hat{c}}}
\newcommand{\ehat}{\bm{\hat{e}}}
\newcommand{\bbar}{\overline{\bm{b}}}
\newcommand{\xhat}{\bm{\hat{x}}}
\newcommand{\yhat}{\bm{\hat{y}}}
\newcommand{\zhat}{\bm{\hat{z}}}

\newcommand{\Xbar}{\bar{\vec{X}}}
\newcommand{\phat}{\bm{\hat{\perp}}}
\newcommand{\that}{\bm{\hat{\theta}}}

\newcommand{\eI}{\bm{\hat{e}}_1}
\newcommand{\eII}{\bm{\hat{e}}_2}
\newcommand{\ud}{\mathrm{d}}

%Derivatives etc.
\newcommand{\pfrac}[2]{\frac{\partial#1}{\partial#2}}
\newcommand{\ffrac}[2]{\frac{\delta#1}{\delta#2}}
\newcommand{\fixd}[1]{\Big{\arrowvert}_{#1}}
\newcommand{\curl}[1]{\nabla \times #1}

\newcommand{\np}{\vec{\nabla}_{\perp}}
\newcommand{\npc}{\nabla_{\perp} \cdot }
\newcommand{\nc}{\vec\nabla\cdot}
\newcommand{\cn}{\cdot\vec\nabla}
\newcommand{\vn}{\vec{\nabla}}
\newcommand{\npar}{\nabla_\parallel}

\newcommand{\GAI}{\Gamma_{1}^{\dagger}}
\newcommand{\GAII}{\Gamma_{1}^{\dagger -1}}
\newcommand{\T}{\mathrm{T}}
\newcommand{\Tp}{\mathcal T^+_{\Delta\varphi}}
\newcommand{\Tm}{\mathcal T^-_{\Delta\varphi}}
\newcommand{\Tpm}{\mathcal T^\pm_{\Delta\varphi}}
\newcommand{\Tdp}{\mathcal T^+_{\delta\varphi}}
\newcommand{\Tdm}{\mathcal T^-_{\delta\varphi}}
\newcommand{\Tdpm}{\mathcal T^\pm_{\delta\varphi}}
%%%%%%%%Some useful abbreviations %%%%%%%%%%%%%%%%
\def\feltor{{\textsc{Feltor }}}

\def\fixme#1{\typeout{FIXME in page \thepage :{#1}}%
 \textsc{\color{red}[{#1}]}}


\usepackage{minted}
\renewcommand{\ne}{\ensuremath{{n_e} }}
\renewcommand{\ni}{\ensuremath{{N_i} }}

%%%%%%%%%%%%%%%%%%%%%%%%%%%%%DOCUMENT%%%%%%%%%%%%%%%%%%%%%%%%%%%%%%%%%%%%%%%
\begin{document}

\title{The toefl project}
\author{ M.~Wiesenberger and M.~Held}
\maketitle

\begin{abstract}
  This is a program for 2d isothermal blob simulations used in References~\cite{Wiesenberger2014,Kube2016,Wiesenberger2017a}.
\end{abstract}
\tableofcontents

\section{Compilation and useage}
The program toefl.cpp can be compiled three ways with
\begin{verbatim}
make <toefl toefl_hpc toefl_mpi> device = <cpu omp gpu>
\end{verbatim}
Run with
\begin{verbatim}
path/to/toefl/toefl input.json
path/to/toefl/toefl_hpc input.json output.nc
echo np_x np_y | mpirun -n np_x*np_y path/to/toefl/toefl_mpi\
    input.json output.nc
\end{verbatim}
All programs write performance informations to std::cout.
The first is for shared memory systems (CPU/OpenMP/GPU) and opens a terminal window with life simulation results.
 The
second is shared memory systems and uses serial netcdf
to write results to a file.
For distributed
memory systems (MPI+CPU/OpenMP/GPU) the program expects the distribution of processes in the
x and y directions as command line input parameters. Also there serial netcdf is used.

\section{The input file}
Input file format: \href{https://en.wikipedia.org/wiki/JSON}{json}

\subsection{Model equations}
Currently we implemented $5$ slightly different sets of equations. $n$ is the electron density, $N$ is the ion gyrocentre density, $\rho$
the vorticity density and $\phi$ is the electric potential. We
use Cartesian coordinates $x$, $y$.

\subsubsection{Delta-f gyrofluid model}
\begin{subequations}
\begin{align}
 -\nabla^2 \phi =  \Gamma_1 N -n, \quad
\psi = \Gamma_1 \phi \quad \Gamma_1 = ( 1- 0.5\tau\nabla^2)^{-1} \\
 \frac{\partial n}{\partial t}     =
    \{ n, \phi\}
  + \kappa \frac{\partial \phi}{\partial y}
  -\kappa \frac{\partial n}{\partial y}
  + \nu \nabla^2 n  \\
  \frac{\partial N}{\partial t} =
  \{ N, \psi\}
  + \kappa \frac{\partial \psi}{\partial y}
  + \tau \kappa\frac{\partial N}{\partial y} +\nu\nabla^2N
\end{align}
\end{subequations}
This model is chosen in the input file via
\begin{minted}[texcomments]{js}
"model":
{
    "type"  : "local",
    "curvature" : 0.0005, // $\kappa$
    "tau" : 0, // $\tau$
    "nu" : 1e-9
}
\end{minted}

\subsubsection{Full-F gyrofluid model}
\begin{subequations}
\begin{align}
B(x)^{-1} = \kappa x +1-\kappa X\quad \Gamma_1 = ( 1- 0.5\tau\nabla^2)^{-1}\\
 -\nabla\cdot \left(\frac{N}{B^2} \nabla_\perp \phi\right) = \Gamma_1 N-n, \quad
 \text{Boussinesq:}\quad -\nabla_\perp^2 \phi = \frac{B^2}{N} (\Gamma_1 N -n) \\
\psi = \Gamma_1 \phi - \frac{1}{2} \frac{(\nabla\phi)^2}{B^2}\\
 \frac{\partial n}{\partial t}     =
    \frac{1}{B}\{ n, \phi\}
  + \kappa n\frac{\partial \phi}{\partial y}
  -\kappa \frac{\partial n}{\partial y}
  + \nu \nabla_\perp^2 n  \\
  \frac{\partial N}{\partial t} =
  \frac{1}{B}\{ N, \psi\}
  + \kappa N\frac{\partial \psi}{\partial y}
  + \tau \kappa\frac{\partial N}{\partial y} +\nu\nabla_\perp^2N
\end{align}
\end{subequations}
This model is chosen in the input file via
\begin{minted}[texcomments]{js}
"model":
{
    "type"  : "global",
    "boussinesq" : false, // or true
    "curvature" : 0.0005, // $\kappa$
    "tau" : 0, // $\tau$
    "nu" : 1e-9
}
\end{minted}

\subsubsection{Gravity delta-f model}
\begin{subequations}
\begin{align}
 \nabla^2 \phi = \rho \\
 \frac{\partial n}{\partial t} = \{ n, \phi\} + \nu \nabla^2 n  \\
  \frac{\partial \rho}{\partial t} = \{ \rho, \phi\} - \eta \rho - \frac{\partial n}{\partial y} + \nu \nabla^2 \rho
\end{align}
\end{subequations}
This model is chosen in the input file via
\begin{minted}[texcomments]{js}
"model":
{
    "type"  : "gravity-local",
    "friction" : 0 // $\eta$
    "nu" : 1e-9
}
\end{minted}


\subsubsection{Gravity full-f model}
\begin{subequations}
\begin{align}
 \nabla \cdot(n \nabla \phi) = \rho \quad\text{ Boussinesq: }\quad \nabla^2 \phi = \rho/n \\
 \frac{\partial n}{\partial t} = \{ n, \phi\} +  \nu \nabla^2 n  \\
  \frac{\partial \rho}{\partial t} = \{ \rho, \phi\} + \{n, \frac{1}{2} \nabla\phi^2\} - \eta \rho - \frac{\partial n}{\partial y} +\nu\nabla^2\rho
\end{align}
\end{subequations}
This model is chosen in the input file via
\begin{minted}[texcomments]{js}
"model":
{
    "type"  : "gravity-global",
    "friction" : 0 // $\eta$
    "nu" : 1e-9
}
\end{minted}

\subsubsection{Full-f global drift-fluid model}
\begin{subequations}
\begin{align}
B(x)^{-1} = \kappa x +1-\kappa X\\
 \nabla \cdot \left(\frac{n}{B^2} \nabla \phi\right) = \rho \quad
 \text{Boussinesq:}\quad \nabla^2\phi = \rho \frac{B^2}{n} \quad
\psi = \frac{1}{2} \frac{(\nabla\phi)^2}{B^2}\\
 \frac{\partial n}{\partial t}     =
    \frac{1}{B}\{ n, \phi\}
  + \kappa n\frac{\partial \phi}{\partial y}
  + \nu \nabla^2 n  \\
  \frac{\partial \rho}{\partial t} =
  \frac{1}{B}\{ \rho, \phi\}
  + \frac{1}{B}\{n, \psi\}
  + \kappa \rho\frac{\partial \phi}{\partial y}
  + \kappa n\frac{\partial \psi}{\partial y}
  - \kappa\frac{\partial n}{\partial y} +\nu\nabla^2\rho 
\end{align}
\end{subequations}
This model is chosen in the input file via
\begin{minted}[texcomments]{js}
"model":
{
    "type"  : "drift-global",
    "boussinesq" : false, // or true
    "curvature" : 0.0005, // $\kappa$
    "nu" : 1e-9
}
\end{minted}
%%%%%%%%%%%%%%%%%%%%%%%%%
\subsection{Spatial grid} \label{sec:spatial}
The spatial grid is an equidistant discontinuous Galerkin discretization of the
2D Cartesian product-space
$[ 0, l_x]\times [0, l_y]$,
We use an equal number of Gaussian nodes in $x$ and $y$.
\begin{minted}[texcomments]{js}
"grid" :
{
    "n"  :  3, // The number of Gaussian nodes in x and y (3 is a good value)
    "Nx"  : 48, // Number of cells in x
    "Ny"  : 48, // Number of cells in y
    "lx"  : 200, // $l_x$
    "ly"  : 200 // $l_y$
}
\end{minted}


\subsection{Initialization}
There only is one initialization type available, namely the blob initial condition.
\begin{align} \label{eq:profile_blob}
    \ne(x,y) &= 1 + A\exp\left( -\frac{(x-X)^2 + (y-Y)^2}{2\sigma^2}\right) \\
\end{align}
where $X = p_xl_x$ and $Y= p_yl_y$ are the initial centre of mass position coordinates, $A$ is the amplitude and $\sigma$ the
radius of the blob.
\begin{minted}[texcomments]{js}
"init":
{
    "amplitude": 1.0, // $A$ in Eq.\eqref{eq:profile_blob}
    "posX" : 0, // $p_x$ in Eq.\eqref{eq:profile_blob}
    "posY" : 0, // $p_y$ in Eq.\eqref{eq:profile_blob}
    "sigma" : 5.0, // $\sigma$ in Eq.\eqref{eq:profile_blob}
}
\end{minted}
In the case of the "local" and "global" models we can choose how to initialize
the ion density
\begin{align}
    \ni(x,y) &= \ne(x,y) \text{ no FLR correction} \\
    \ni(x,y) &= \Gamma_{1i}^{-1} \ne(x,y) \text{ inverse FLR correction} \\
\end{align}
\begin{minted}[texcomments]{js}
"init":
{
    "flr" : "none", //only needed for $\tau_i \neq 0$
    "flr" : "gamma_inv", //only needed for $\tau_i \neq 0$
}
\end{minted}
In the case of the "gravity\_local" and "gravity\_global" and "drift\_global" models
the vorticity is initialized to zero:
\begin{align}
    \rho(x,y) = 0
\end{align}

\subsection{Timestepper}
We use an adaptive explicit embedded Runge Kutta timestepper to advance the equations in time
\begin{minted}[texcomments]{js}
"timestepper":
{
    "tableau" : "Bogacki-Shampine-4-2-3",
    "rtol" : 1e-5,
    "atol" : 1e-6
}
\end{minted}
\subsection{Boundary conditions}
The boundary conditions are the same for all fields.
\begin{minted}[texcomments]{js}
"bc" : ["DIR", "PER"]
\end{minted}

\subsection{Elliptic solvers}
We discretize all elliptic operators with a local dG method (LDG).  In order to
solve the elliptic equations we chose a multigrid scheme (nested iterations in
combination with conjugate gradient solves on each plane). The accuaracies for
the polarization equation can be chosen for each stage separately, while for
the Helmholtz type equations (the gamma operators) only
one accuracy can be set (they typically are quite fast to solve):
\begin{minted}[texcomments]{js}
"elliptic":
{
    "stages"    : 3,  // Number of stages (3 is best in virtually all cases)
    // $2^{\text{stages-1}}$ has to evenly divide both $N_x$ and $N_y$
    "eps_pol"   : [1e-6,10,10],
    // The first number is the tolerance for residual of the inversion of
    // polarisation equation. The second number is a multiplicative
    // factor for the accuracy on the second grid in a multigrid scheme, the
    // third for the third grid and so on:
    // $\eps_0 = \eps_{pol,0}$, $\eps_i = \eps_{pol,i} \eps_{pol,0}$  for $i>1$.
    // Tuning those factors is a major performance tuning oppourtunity!!
    // For saturated turbulence the suggested values are [1e-6, 2000, 100].
    "eps_gamma" : [1e-10,1,1] // Accuracy requirement of Gamma operator
    "direction" : "forward", // Direction of the Laplacian: forward or centered
}
\end{minted}
\begin{tcolorbox}[title=Note]
    We use solutions from previous time steps to extrapolate an initial guess
    and use a diagonal preconditioner
\end{tcolorbox}
\section{Output}
Our program can either write results directly to screen using the glfw library
or write results to disc using netcdf.
This can be controlled via
\begin{minted}[texcomments]{js}
"output":
{
    // Use glfw to display results in a window while computing (requires to
    // compile with the glfw3 library)
    "type"  : "glfw"
    "itstp"  : 4, // The number of steps between refreshing field plots

    // Use netcdf to write results into a file (filename given on command line)
    // (see next section for information about what is written in there)
    "type"  : "netcdf"
    "tend"  : 4,    // The number of steps between outputs of 2d fields
    "maxout"  : 100, // The total number of field outputs
    "n"  : 3 , // The number of polynomial coefficients in the output file
    "Nx" : 48, // Number of cells in x in the output file
    "Ny" : 48  // Number of cells in y in the output file
}
\end{minted}
The number of points in the output file can be lower (or higher) than the number of
grid points used for the calculation. The points will be interpolated from the
computational grid.
\subsection{Netcdf file}
Output file format: netcdf-4/hdf5

\begin{longtable}{lll>{\RaggedRight}p{7cm}}
\toprule
\rowcolor{gray!50}\textbf{Name} &  \textbf{Type} & \textbf{Dimension} & \textbf{Description}  \\ \midrule
inputfile        & text attribute & 1 & verbose input file as a string \\
time             & Coord. Var. & 1 (time) & time at which fields are written \\
x                & Coord. Var. & 1 (x) & x-coordinate  \\
y                & Coord. Var. & 1 (y) & y-coordinate \\
xc               & Dataset & 2 (y,x) & Cartesian x-coordinate  \\
yc               & Dataset & 2 (y,x) & Cartesian y-coordinate \\
weights          & Dataset & 2 (y,x) & Gaussian integration weights \\
X                & Dataset & 3 (time, y, x) & 2d outputs \\
\bottomrule
\end{longtable}
The output fields X are determined in the file \texttt{toefl/diag.h}.

%%%%%%%%%%%%%%%%%%%%%%%%%%%%%%%%
\section{Blob related quantities}
For the blob quantities it is important that we use the physical $\ne$.
\begin{tcolorbox}[title=Note]
    These quantities can be easily computed in post-processing in e.g. a python script.
\end{tcolorbox}

\subsection{Center of mass}
The first quantity of interest is the Center-of-mass position
\begin{align}
    M :=& \int (\ne -1 ) \dV \\
    X :=& \frac{1}{M} \int x(\ne - 1) \dV \\
    Y :=& \frac{1}{M} \int y(\ne - 1) \dV
\end{align}
\subsection{Blob compactness}
The blobs ability to retain its initial (Gaussian) shape is quantified by the  blob compactness
\begin{align}
     I_c(t) &:= \frac{\int dA (\ne(\vec{x},t)-1) h(\vec{x},t)}{\int dA
(\ne(\vec{x},0)-1) h(\vec{x},0)}
\end{align}
Here, we introduced the heaviside function
\begin{align}
     h(\vec{x},t) &:= \begin{cases}
          1,
        &\ \text{if} \hspace{2mm}||\vec{x} - \vec{X}_{max}||^2 < \sigma^2 \\
0,  &\ \text{else}
           \end{cases} \nonumber
\end{align}
and the position of the maximum density \( \vec{X}_{max}(t)\).


%%%%%%%%%%%%%%%%%%%%%%%%%%%%%%%%%%%%%%%
\section{Invariants}
\subsection{Mass}
In all models the particles density conservation reads
\begin{align} \label{eq:mass_theorem}
  \frac{\partial}{\partial t} \mathcal M
  + \nc\vec{ j_{n,e}}
  =  \Lambda_{\ne}
\end{align}
with
\begin{align}
    \mathcal M &= \ne \\
    \vec j_{n,s} &= \ne \vec v_E + \vec v_\kappa \\
     \Lambda_\ne &= -\nu_\perp \Delta_\perp^2 \ne
\end{align}
With vanishing flux on the boundaries and zero viscosity we have
\begin{align}
    \frac{\partial}{\partial t} \int \dV \mathcal M_s = 0
\end{align}

\subsection{Energy}
The energy theorem reads
\begin{align} \label{eq:energy_theorem}
  \frac{\partial}{\partial t} \mathcal E
  + \nc\vec{ j_{\mathcal E}}
  =  \Lambda_{\mathcal E}
\end{align}
With our choice of boundary conditions the energy flux
vanishes on the boundary.
In the absence of artificial viscosity the volume integrated
 energy density is thus an exact invariant of the system.
\begin{align}
    \frac{\partial}{\partial t} \int \dV \mathcal E = 0
\end{align}

\subsubsection{ Full-F gyro-fluid model}
The inherent energy density of our system is:
\begin{align}
 \mathcal{E} := &
 \ne \ln{(\ne)} + \tau_i \ni\ln{(\ni)}  + \frac{1}{2} \ni u_E^2
\end{align}
The energy current density and diffusion are
\begin{align}
  \vec j_{\mathcal E} =& \left[
  \left( \ln \ne - \phi \right)\ne\left(
  \vec u_E + \vec u_\kappa \right) \right]
  +\left[
      \left(\tau_i \ln \ni + \psi_{i} \right)\ni\left(
  \vec u_E^i + \vec u_\kappa \right) \right]
  , \\
    \Lambda_\mathcal{E} :=  &+ \left( ( 1+\ln \ne) - \phi\right)(\nu_\perp \Delta_\perp \ne)+ \left( ( 1+\ln \ni) + \psi\right)(\nu_\perp \Delta_\perp \ni)
    \label{eq:energy_diffusion}
\end{align}
where in the energy flux $\vec j_{\mathcal E}$
we neglect terms  containing time derivatives
of the eletric potentials and we sum over all species.

With our choice of boundary conditions the energy flux
vanishes on the boundary.
In the absence of artificial viscosity the volume integrated
 energy density is thus an exact invariant of the system.
\begin{align}
    \frac{\partial}{\partial t} \int \dV \mathcal E = 0
\end{align}
\subsubsection{ Delta-F gyro-fluid model}
The inherent energy density of our system is:
\begin{align}
 \mathcal{E} := &
 \frac{1}{2}\ne^2  + \frac{1}{2}\tau_i \ni^2 + \frac{1}{2} (\nabla\phi)^2
\end{align}
The energy current density and diffusion are
\begin{align}
  \vec j_{\mathcal E} =& \sum_s a_s\left[
  \left(\tau_s \ln \ne + \psi_s \right)\ne\left(
  \vec u_E + \vec u_\kappa \right) \right]
  , \\
    \Lambda_\mathcal{E} :=  &+ \left( ( 1+\ln \ne) - \phi\right)(\nu_\perp \Delta_\perp \ne)
    + \left( \tau_i( 1+\ln \ni) + \psi\right)(\nu_\perp \Delta_\perp \ni)
    \label{eq:energy_diffusion}
\end{align}
where in the energy flux $\vec j_{\mathcal E}$
we neglect terms  containing time derivatives
of the eletric potentials and we sum over all species.

\subsubsection{Gravity delta-f model}
\begin{align}
 \mathcal{E} := &
 \frac{1}{2} \ne^2
\end{align}

\subsubsection{Gravity full-f model}
\begin{align}
 \mathcal{E} := &
 \frac{1}{2} \ne^2
\end{align}

\subsubsection{Full-f global drift-fluid model}
\begin{align}
 \mathcal{E} := & n \ln ( n) + \frac{1}{2} n u_E^2
\end{align}



%..................................................................
\bibliography{../common/references}
%..................................................................

\end{document}
