%%%%%%%%%%%%%%%%%%%%%definitions%%%%%%%%%%%%%%%%%%%%%%%%%%%%%%%%%%%%%%%

%\documentclass[12pt]{article}
%\documentclass[12pt]{scrartcl}
\documentclass{hitec} % contained in texlive-latex-extra
\settextfraction{0.9} % indent text
\usepackage{csquotes}
\usepackage[hidelinks]{hyperref} % doi links are short and usefull?
\hypersetup{%
    colorlinks=true,
    linkcolor=blue,
    urlcolor=magenta
}
\urlstyle{rm}
\usepackage[english]{babel}
\usepackage{mathtools} % loads and extends amsmath
\usepackage{amssymb}
% packages not used
%\usepackage{graphicx}
%\usepackage{amsthm}
%\usepackage{subfig}
\usepackage{bm}
\usepackage{longtable}
\usepackage{booktabs}
\usepackage{ragged2e} % maybe use \RaggedRight for tables and literature?
\usepackage[table]{xcolor} % for alternating colors
%\rowcolors{2}{gray!25}{white}
\renewcommand\arraystretch{1.3}
\usepackage{doi}
\usepackage[sort,square,numbers]{natbib}
\bibliographystyle{abbrvnat}

\definecolor{light-gray}{gray}{0.95}
\newcommand{\code}[1]{\colorbox{light-gray}{\texttt{#1}}}
\newcommand{\eps}{\varepsilon}
\renewcommand{\d}{\mathrm{d}}
\renewcommand{\vec}[1]{{\boldsymbol{#1}}}
\newcommand{\dx}{\,\mathrm{d}x}
%\newcommand{\dA}{\,\mathrm{d}(x,y)}
%\newcommand{\dV}{\mathrm{d}^3{x}\,}
\newcommand{\dA}{\,\mathrm{dA}}
\newcommand{\dV}{\mathrm{dV}\,}

\newcommand{\Eins}{\mathbf{1}}

\newcommand{\ExB}{$\bm{E}\times\bm{B} \,$}
\newcommand{\GKI}{\int d^6 \bm{Z} \BSP}
\newcommand{\GKIV}{\int dv_{\|} d \mu d \theta \BSP}
\newcommand{\BSP}{B_{\|}^*}
\newcommand{\Abar}{\langle A_\parallel \rangle}
%Averages
\newcommand{\RA}[1]{\left \langle #1 \right \rangle} %Reynolds (flux-surface) average
\newcommand{\RF}[1]{\widetilde{#1}} %Reynolds fluctuation
\newcommand{\FA}[1]{\left[\left[ #1 \right]\right]} %Favre average
\newcommand{\FF}[1]{\widehat{#1}} %Favre fluctuation
\newcommand{\PA}[1]{\left \langle #1 \right\rangle_\varphi} %Phi average

%Vectors
\newcommand{\ahat}{\bm{\hat{a}}}
\newcommand{\bhat}{\bm{\hat{b}}}
\newcommand{\chat}{\bm{\hat{c}}}
\newcommand{\ehat}{\bm{\hat{e}}}
\newcommand{\bbar}{\overline{\bm{b}}}
\newcommand{\xhat}{\bm{\hat{x}}}
\newcommand{\yhat}{\bm{\hat{y}}}
\newcommand{\zhat}{\bm{\hat{z}}}

\newcommand{\Xbar}{\bar{\vec{X}}}
\newcommand{\phat}{\bm{\hat{\perp}}}
\newcommand{\that}{\bm{\hat{\theta}}}

\newcommand{\eI}{\bm{\hat{e}}_1}
\newcommand{\eII}{\bm{\hat{e}}_2}
\newcommand{\ud}{\mathrm{d}}

%Derivatives etc.
\newcommand{\pfrac}[2]{\frac{\partial#1}{\partial#2}}
\newcommand{\ffrac}[2]{\frac{\delta#1}{\delta#2}}
\newcommand{\fixd}[1]{\Big{\arrowvert}_{#1}}
\newcommand{\curl}[1]{\nabla \times #1}

\newcommand{\np}{\vec{\nabla}_{\perp}}
\newcommand{\npc}{\nabla_{\perp} \cdot }
\newcommand{\nc}{\vec\nabla\cdot}
\newcommand{\cn}{\cdot\vec\nabla}
\newcommand{\vn}{\vec{\nabla}}
\newcommand{\npar}{\nabla_\parallel}

\newcommand{\GAI}{\Gamma_{1}^{\dagger}}
\newcommand{\GAII}{\Gamma_{1}^{\dagger -1}}
\newcommand{\T}{\mathrm{T}}
\newcommand{\Tp}{\mathcal T^+_{\Delta\varphi}}
\newcommand{\Tm}{\mathcal T^-_{\Delta\varphi}}
\newcommand{\Tpm}{\mathcal T^\pm_{\Delta\varphi}}
\newcommand{\Tdp}{\mathcal T^+_{\delta\varphi}}
\newcommand{\Tdm}{\mathcal T^-_{\delta\varphi}}
\newcommand{\Tdpm}{\mathcal T^\pm_{\delta\varphi}}
%%%%%%%%Some useful abbreviations %%%%%%%%%%%%%%%%
\def\feltor{{\textsc{Feltor }}}

\def\fixme#1{\typeout{FIXME in page \thepage :{#1}}%
 \textsc{\color{red}[{#1}]}}


\usepackage{minted}
\renewcommand{\ne}{\ensuremath{{n_e} }}
\renewcommand{\ni}{\ensuremath{{N_i} }}

%%%%%%%%%%%%%%%%%%%%%%%%%%%%%DOCUMENT%%%%%%%%%%%%%%%%%%%%%%%%%%%%%%%%%%%%%%%
\begin{document}

\title{Diagnostics of geometry}
\author{ M.~Wiesenberger and M.~Held}
\maketitle

\begin{abstract}
  This is a program for 1d, 2d and 3d diagnostics of magnetic field geometry in Feltor
\end{abstract}
\tableofcontents

\section{Compilation and useage}
The program geometry\_diag.cpp can be compiled with
\begin{verbatim}
make geometry_diag
\end{verbatim}
Run with
\begin{verbatim}
path/to/geometry_diag/geometry_diag input.json output.nc
\end{verbatim}
The program writes performance informations to std::cout and uses serial netcdf to write a netcdf file.

\section{The input file}
Input file format: \href{https://en.wikipedia.org/wiki/JSON}{json}

%%%%%%%%%%%%%%%%%%%%%%%%%
\subsection{Spatial grid} \label{sec:spatial}
The spatial grid is an equidistant discontinuous Galerkin discretization of the
3D Cylindrical product-space
$[ R_{\min}, R_{\max}]\times [Z_{\min}, Z_{\max}] \times [0,2\pi]$,
where we define
\begin{align} \label{eq:box}
    R_{\min}&=R_0-\varepsilon_{R-}a\quad
    &&R_{\max}=R_0+\varepsilon_{R+}a\nonumber\\
    Z_{\min}&=-\varepsilon_{Z-}a\quad
    &&Z_{\max}=\varepsilon_{Z+}a
\end{align}
We use an equal number of Gaussian nodes in $x$ and $y$.
\begin{minted}[texcomments]{js}
"grid" :
{
    "n"  :  3, // The number of Gaussian nodes in x and y (3 is a good value)
    "Nx"  : 48, // Number of cells in R
    "Ny"  : 48, // Number of cells in Z
    "Nz"  : 20, // Number of cells in varphi
    "scaleR"  : [1.1,1.1], // $[\varepsilon_{R-}, \varepsilon_{R+}]$ scale left and right boundary in R in Eq.\eqref{eq:box}\\
    "scaleZ"  : [1.2,1.1], // $[\varepsilon_{Z-}, \varepsilon_{Z+}]$ scale lower and upper boundary in Z in Eq.\eqref{eq:box}
}
\end{minted}
Furthermore, a user can specify which kind of diagnostics quantities should be computed and how:
\begin{minted}[texcomments]{js}
"grid" :
{
    // for flux surface average computations
    "Npsi": 64, // resolution of X-point grid for fsa
    "Neta": 640, // resolution of X-point grid for fsa
    "fx_0" : 0.125, // where the separatrix is in relation to the $\zeta$ coordinate
}
\end{minted}
\subsection{The boundary region} \label{sec:boundary}
Setting the boundary conditions in an appropriate manner is probably the most
fiddly task in setting up a simulation.
%%%%%%%%%%%%%%%%%%%%%%%%%%%%%%%%%%%%%%%%%%%%%%%
\subsubsection{The wall region}\label{sec:wall}
\begin{minted}[texcomments]{js}
"boundary":
{
    "wall" :
    {
        // no wall region
        "type" : "none"
    }
}
\end{minted}
Being a box, our computational domain is in particular not aligned with the
magnetic flux surfaces. This means that particularly in the corners of
the domain the field lines inside the domain are very short (in the
sense that the distance between the entry point and leave point is short).
It turns out that this behaviour is numerically disadvantageous (may
blow up the simulation in the worst case) in the
computation of parallel derivatives.
In order to remedy this situation
we propose a penalization method to model the actual physical wall.
We define an approximation to the step function with a transition layer of radius $a$
around the origin
\begin{align}
\Theta_a(x) := \begin{cases}
    0 & \text{ for } x \leq -a  \\
    \frac{1}{32 a^7}  \left(16 a^3-29 a^2 x+20 a x^2-5 x^3\right) (a+x)^4
    &\text{ for } -a<x\leq a \\
    1 & \text{ for } x > a
\end{cases}
    \approx H(x)
\label{eq:approx_heaviside}
\end{align}
where $H(x)$ is the Heaviside step function.

We now use the region defined by
\begin{align}\label{eq:wall}
    \chi_w(R,Z,\varphi):=\Theta_{\alpha/2}\left(\psi_{p,b} + \frac{\alpha}{2} - \psi \right) \approx H(\psi_{p,b}-\psi)
\end{align}
to define the wall region.
In order to simplify the setup of this region we give $\psi_{p,b}$ and $\alpha$ in terms of
$\rho_p$ and $\alpha_p$ via $\psi_{p,b} = (1-\rho_{p,b}^2)\psi_{p,O}$ and $\alpha = -(2\rho_{p,b} \alpha_p + \alpha_p^2)\psi_{p,O}$. In case we change the sign
of $\psi_p$ via $\mathcal P_\psi$ (to make it concave) note that $\alpha$ becomes
negative and $\psi_{p,O}$ is positive).
We then need to point mirror Eq.~\eqref{eq:wall} at $\psi_{p,b}+\frac{\alpha}{2}$.

\begin{minted}[texcomments]{js}
"boundary":
{
    "wall" :
    {
        // Simple flux aligned wall above a threshold value
        "type" : "heaviside",
        "boundary" : 1.2, // wall region boundary $\rho_{p,b}$
        // yields $\psi_0 = (1-\rho_{p,b}^2)\psi_{p,O}$ in Eq.\eqref{eq:wall}.
        "alpha" : 0.25, // Transition width $\alpha_p$: yields
        // $\alpha=-2\rho_{p,b}\alpha_p+\alpha_p^2)\psi_{p,O}$ for the Heaviside
        // in the wall function \eqref{eq:wall}.

        // Double flux aligned wall above and below a threshold value
        "type" : "sol_pfr",
        "boundary" : [1.2, 0.8],
        "alpha" : [0.25,0.25],
        // first one is for main region, second one for PFR
    }
}
\end{minted}
\noindent

%%%%%%%%%%%%%%%%%%%%%%%%%%%%%%%%%%%%%%%%%%%%%%%%%%%
\subsubsection{The sheath region}\label{sec:sheath}
\begin{minted}[texcomments]{js}
"boundary":
{
    "sheath" :
    {
        // no sheath region
        "type" : "none",
    }
}
\end{minted}
In order to define sheath boundary conditions we first define a sheath region
and then determine whether the field lines point toward the wall or away from it.
We define as sheath any part on the bounding box that is not included in the wall
penalization. Then we check for each point in the box the poloidal distance (in
terms of angle $\varphi$) to the sheath wall and if the poloidal field points
toward or away from the wall closest to it.
We then take $\theta_{\alpha/2}\left( 2\pi(\eps_s - \frac{\alpha}{2}) - d(R,Z)\right)$
and take the set intersection of that region and the ``not wall'' region to
determine the sheath penalization region:
\begin{align}\label{eq:sheath}
\chi_s := \left(1-\chi_w(R,Z,\varphi)\right) \theta_{\alpha/2}\left( 2\pi\left(\eps_s - \frac{\alpha}{2}\right) - d(R,Z)\right)
\end{align}
\begin{minted}[texcomments]{js}
"boundary" :
{
    "sheath" :
    {
        "boundary" : 3/32, // Total width of the sheath $\eps_s$ away from the wall
        // in units of $2\pi$ in Eq.\eqref{eq:sheath}
        "alpha" : 2/32, // Transition width $\alpha$
        // in units of $2\pi$ in Eq.\eqref{eq:sheath}.
        "max_angle" : 4 // $\varphi_{\max}$ in units of $2\pi$
        // in order to compute field-line following coordinates
        // we need to integrate fieldlines. In order to avoid infinite integration
        // we here give a maximum angle where to stop integration
    }
}
\end{minted}
\noindent

\subsection{Diagnostics}

By default only the standard basic diagnostics is run that has little danger of failing.
ALL of the following can fail and should only be attempted once the parameters are tuned!
The diagnostics field is a list of flags that can be empty
\begin{minted}[texcomments]{js}
"diagnostics" : ["q-profile", // integrate field-lines within LCFS to get q-profile (
     //can fail e.g. if Psi\_p = 0 is not a closed flux-surface)
"fsa", //compute a flux-aligned grid and compute flux-surface averaged quantities
"sheath" // integrate field-lines to get distance to divertor
\end{minted}

\subsection{Magnetic field} \label{sec:geometry_file}
The json structure of the geometry parameters depends on which expansion for $\psi_p$ is chosen Eq.~\eqref{eq:solovev} or Eq.~\eqref{eq:polynomial}.
In addition we have an option to read the geometry parameters either from an external
file or directly from a field in the input file.
\begin{minted}[texcomments]{js}
"magnetic_field":
{
    // Tells the parser that the geometry parameters are located in an
    // external file the json file (relative to where the program is
    //  executed) containing the geometry parameters to read
    "input": "file",
    "file": "path/to/geometry.json",
    //
    // Tells the parser that the geometry parameters are located in the
    // same file in the params field (recommended option)
    "input": "params",
    "params":
    // copy of the geometry.json
}
\end{minted}
\noindent
\subsection{q-profile}

In the computation of the q-profile using the delta-function method the delta function can have
a width parameter
\begin{minted}[texcomments]{js}
"width-factor" : 0.03
\end{minted}


%%%%%%%%%%%%%%%%%%%%%%%%%%%%%%%%

%..................................................................
\bibliography{../common/references}
%..................................................................

\end{document}
